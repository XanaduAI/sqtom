%\documentclass[9pt,letterpaper,aps,pra,notitlepage,twocolumn]{revtex4-1}
\documentclass[aps,twocolumn,reprint,nofootinbib]{revtex4-1}
\usepackage{amsmath}
\usepackage{amssymb}
\newcommand{\pr}[1]{| #1 \rangle \langle #1|}
\newcommand{\ket}[1]{| #1 \rangle}
\newcommand{\bra}[1]{\langle #1 |}
\newcommand{\braket}[1]{\langle #1 \rangle}
\newcommand{\ketq}[1]{| #1 _q\rangle }
\newcommand{\braq}[1]{\langle #1 _q |}
\newcommand{\hc}{\text{H.c.}}
\newcommand{\cc}{\text{c.c.}}
\newcommand{\h}{\hat}
\newcommand{\eq}[1]{\begin{align}#1\end{align}}
\newcommand{\nn}{\nonumber}
\renewcommand{\a}{\hat a}
\newcommand{\ad}{\hat a^\dagger}
\newcommand{\Ai}{\text{Ai}}
\newcommand{\Aip}{\text{Ai}'}

\begin{document}
	
	\title{The $g^{(2)}$ of noisy squeezed light}
	\author{Nicol\'as Quesada}
	\affiliation{Xanadu, Toronto, Canada}
	\date{\today}
	\maketitle

\maketitle

We are interested in calculating the second order coherence function at zero delay 
\eq{\label{eq:def}
g^{(2)} = \frac{\braket{N^2-N}}{\braket{N}^2},
}
for a beam that contains a number of single mode squeezed vacuum states and a ``dark mode'' with Poisson number statistics.
We use the index 0 for the dark mode and Latin indices for the squeezed modes starting at 1. Finally, we use Greek indices for the dark mode together with the squeezed modes thus,
\eq{
N = n_0+\sum_{i=1} n_i = \sum_{\mu=0} n_\mu.	
}
Since the dark mode has Poisson statistics then
\eq{
\braket{n_0} &= \bar{n}_0, \quad \braket{n_0^2} = \bar{n}_0^2+\bar{n}_0.
}
%Similarly for the squeezed modes
%\eq{
%\braket{n_i} &= \bar{n}_i \\
%\braket{n_i^2}&= 3	\bar{n}_i^2+2\bar{n}_i.
%}
%For each mode we can associate a destruction operator $a_\mu$ such that $n_\mu = a^\dagger_\mu a_\mu$.
We can then write the denominator in Eq.~\eqref{eq:def} as
\eq{
\braket{N^2-N} &= \braket{\sum_{\mu} n_\mu \sum_{\nu} n_\nu}	- \braket{\sum_{\mu }n_\mu} \\
&=\braket{n_0^2+2n_0 \sum_i n_i +\sum_{i}\sum_j n_i n_j} - \bar{n}_0 - \bar{M},
}
where we introduced $\bar{M} = \sum_i \bar{n}_i = \sum_i \braket{n}_i$ as the total mean photon number of the squeezed modes, in terms of which we can write
\eq{
\braket{N} = \bar{N} =  \bar{n}_0	+ \bar{M}
}

We also assume that the dark mode is uncorrelated with the squeezed modes allowing us to write
\eq{
\braket{N^2-N} &=	\bar{n}_0^2 + 2 \bar{n}_0 \bar{M} + \left\langle \sum_{i}\sum_j n_i n_j \right\rangle - \bar{M},
}
where in the last line we used the fact that $\braket{n_0^2} -\bar{n}_0 = \bar{n}_0^2$.
We can finally write the last equation as
\eq{
\braket{N^2-N} = \bar{M}^2 \left( \underbrace{\frac{\braket{\sum_{i}\sum_j n_i n_j} - \bar{M}}{\bar{M}^2}}_{\equiv g^{(2)}_{\text{noiseless}}} + 2 \frac{\bar{n}_0}{\bar M} +\left(\frac{\bar{n}_0}{\bar M} \right)^2 \right).
}
In the last equation we introduced $g^{(2)}_{\text{noiseless}}$ as the second order coherence function if there were no noise photons. 


One can also easily write the denominator of Eq.~\eqref{eq:def} as
\eq{
\braket{N}^2 = \bar{M}^2 \left(1+ 2 \frac{\bar{n}_0}{\bar M} +\left(\frac{\bar{n}_0}{\bar M} \right)^2\right).
}
With this last piece we finally write
\eq{
g^{(2)} = \frac{  g^{(2)}_{\text{noiseless}} + 2 \frac{\bar{n}_0}{\bar M} +\left(\frac{\bar{n}_0}{\bar M} \right)^2}{1+ 2 \frac{\bar{n}_0}{\bar M} +\left(\frac{\bar{n}_0}{\bar M} \right)^2}.	
}

The last equation can expanded in the limit where $n_0/M \ll 1$ to obtain
\eq{
g^{(2)} = g^{(2)}_{\text{noiseless}} -2 \left(g^{(2)}_{\text{noiseless}} -1\right) \frac{n_0}{M}
}
One can now specialize the last equation to two cases. The first one is where on the modes of interest come from a degenerate squeezer
\eq{
	g^{(2)}_{\text{noiseless}} = 1+ \frac{1}{\bar{M}}+\frac{2}{K}.
}
where 
\eq{
K = \frac{\left( \sum_i \bar{n}_i\right)^2}{\sum_i \bar{n}_i^2} \geq 1.		
}
where $K$ is the Schmidt number.
The second case is when one has one half of a twin beam for which one has (cf. Christ et al.)
\eq{
g^{(2)}_{\text{noiseless}} = 1+\frac{1}{K}.
}
Note that in general the amount of noise in the ``signal'' and ``idler'' arm of a twin beam can be different leading to different values for the $g^{(2)}$. In the limit where the noise photon number is much smaller than the twin beam photon number one can write the following equality
\eq{
g^{(2)}_s - g^{(2)}_i = -2 (g^{(2)}_\text{noiseless} -1) \frac{n_s - n_i}{M}
}



\end{document}
