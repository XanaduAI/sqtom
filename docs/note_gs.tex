%\documentclass[9pt,letterpaper,aps,pra,notitlepage,twocolumn]{revtex4-1}
\documentclass[aps,twocolumn,reprint,nofootinbib]{revtex4-1}
\usepackage{amsmath}
\usepackage{amssymb}
\newcommand{\pr}[1]{| #1 \rangle \langle #1|}
\newcommand{\ket}[1]{| #1 \rangle}
\newcommand{\bra}[1]{\langle #1 |}
\newcommand{\braket}[1]{\langle #1 \rangle}
\newcommand{\ketq}[1]{| #1 _q\rangle }
\newcommand{\braq}[1]{\langle #1 _q |}
\newcommand{\hc}{\text{H.c.}}
\newcommand{\cc}{\text{c.c.}}
\newcommand{\h}{\hat}
\newcommand{\eq}[1]{\begin{align}#1\end{align}}
\newcommand{\nn}{\nonumber}
\renewcommand{\a}{\hat a}
\newcommand{\ad}{\hat a^\dagger}
\newcommand{\Ai}{\text{Ai}}
\newcommand{\Aip}{\text{Ai}'}

\begin{document}
\title{Correlation functions and high-gain Klyshko coefficients}
\author{Nicol\'as Quesada}
\affiliation{Xanadu, Toronto, Canada}
\date{\today}
\maketitle
It is common to use the second order correlation function [1]
\eq{
g^2_{J} = \frac{\braket{N_{J}(N_J-1)}}{\braket{N_J}^2},
}
to infer the Schmidt number of twin-beam source. This ratio of photon number moments has the important property of being insensitive to loss. If we denote by $\bar{n}_i$ the mean photon number occupation of the $i^{\text{th}}$ Schmidt mode then the Schmidt number is defined as
\eq{
K = \frac{\left( \sum_i \bar{n}_i  \right)^2}{ \sum_i \bar{n}_i^2}	\geq 1,
}
and it is related to the $g^2$ of the signal or idler as
\eq{
g^2 = g^2_S = g^2_I = 1+\frac{1}{K}	.
}
Another useful measurement is $g^{1,1}$ which uses joint statistics and is given by [1]
\eq{
g^{1,1} = \frac{\braket{N_S N_I}}{\braket{N_S} \braket{N_I}}.
}
This ratio of photon number moments is also independent of the loss in either signal and idler and for a twin-beam source it can be linked to the $g^2$ and the total number of photons at the source $\bar{N} = \sum_{i} \bar{n}_i$, i.e. before any loss occurs, as follows
\eq{
g^{1,1} = \frac{1}{\bar N} + g^{2} \Longrightarrow \bar N  = \frac{1}{g^{1,1} - g^2}.
}
We can use this last equation to infer the losses in the system by writing first the transmission as
\eq{
\eta_S = \frac{\braket{N_S}}{\bar{N}}, \quad \eta_I = \frac{\braket{N_I}}{\bar{N}},
}
and then finding
\eq{\label{klyshko}
%	\begin{subequations}
\eta_S = \frac{g^{1,1}	\braket{N_S}}{1+g^2 \bar{N}} =
 \left( g^{1,1} -g^2 \right) \braket{N_S},\\
\eta_I = \frac{g^{1,1}	\braket{N_I}}{1+g^2 \bar{N}} =
 \left( g^{1,1} -g^2 \right) \braket{N_I}.
%\end{subequations}
}
Note that in the ``pair limit'', $\bar{N} \ll 1$ and since $1 \leq g^2 \leq 2$ one obtains
\eq{
\eta_S &= g^{1,1}	\braket{N_S} = \frac{\braket{N_S N_I}}{\braket{N_I}} = \frac{\text{coincidences}}{\text{idler counts}},	\\
\eta_I &= g^{1,1}	\braket{N_I} = \frac{\braket{N_S N_I}}{\braket{N_S}}	=\frac{\text{coincidences}}{\text{signal counts}},
}	
which are the well known Klyshko coefficients used to estimate the loss when using threshold detectors; the ratios in Eqs.~\eqref{klyshko} are thus a reasonable generalization of the Klyshko coefficients in the high-gain regime.


In the limit where $g^2$ is close to 2 one can assume there are only 2 Schmidt modes and in this case we can invert directly for the Schmidt number occupations
\eq{
\bar{n}_{1/2} &= \frac{1 \pm \sqrt{2g^2-3}}{2 g^{1,1}-2 g^2}
}
Note that this is only valid for $g^2 \geq 3/2$ which is the smallest $g^2$ one can attain with at most two Schmidt modes.



One can also study what happens when one allows for dark counts where the occupation of the dark modes for the signal and idler are $m_S$ and $m_I$. Then one can write the $g$'s as functions of the dark modes occupations as
\eq{
g^2_J(m_J) &= g^2_J(m_J=0) - \frac{2 m_J}{\bar{N}}\left(g^2(m_J =0) - 1\right)\\
g^{1,1}(m_S, m_I) &= g^{1,1}(m_S=0, m_I=0) \\
& \quad - \frac{m_S+m_I}{\bar N} \left( g^{1,1}(m_S=0, m_I=0) -1 \right).\nn
}

\bigskip

[1] Christ et al. Probing multimode squeezing with correlation functions, New Journal of Physics 13, 033027.

\end{document}